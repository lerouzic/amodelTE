\documentclass[10pt,a4paper]{article}
\usepackage[utf8]{inputenc}


%~ \usepackage{cite}
%~ \makeatletter
%~ \renewcommand{\@biblabel}[1]{\quad#1.}
%~ \makeatother


\usepackage{nameref}
\usepackage[colorlinks=true,citecolor={Gray}]{hyperref}
\usepackage[right]{lineno}

\usepackage{url}

\usepackage[top=0.85in,left=2.75in,footskip=0.75in,marginparwidth=2in]{geometry}

% headrule, footrule and page numbers
\usepackage{lastpage,fancyhdr,graphicx}
\usepackage{epstopdf}
\pagestyle{myheadings}
\pagestyle{fancy}
\fancyhf{}
\rfoot{\thepage/\pageref{LastPage}}
\renewcommand{\footrule}{\hrule height 2pt \vspace{2mm}}
\fancyheadoffset[L]{2.25in}
\fancyfootoffset[L]{2.25in}

\raggedright
\setlength{\parindent}{0.5cm}
\textwidth 5.0in 
\textheight 8.75in

\usepackage{changepage}

\usepackage{microtype}
\DisableLigatures[f]{encoding = *, family = * }

\usepackage{color}
\definecolor{Gray}{gray}{.25}


\usepackage{graphicx}
\usepackage{sidecap}
\usepackage{wrapfig}
\usepackage[pscoord]{eso-pic}
\usepackage[fulladjust]{marginnote}
\reversemarginpar

\usepackage[natbib=true, style=authoryear, uniquename=false, maxcitenames=2, maxbibnames=99, firstinits=true, doi=false, url=false]{biblatex}
\bibliography{amodel.bib}
\renewbibmacro{in:}{}                                  % remove In:
\DeclareFieldFormat[article, inbook]{title}{#1}        % remove quotes around titles
\AtEveryBibitem{%                                      % remove language indications
  \clearlist{language}%
}

\usepackage{bm}
\usepackage{amsmath}

\usepackage[english]{babel}



\title{}
\date{}

\begin{document}
\vspace*{0.35in}

% title goes here:
\begin{flushleft}
{\Large
\textbf\newline{A Population Genetics Theory for piRNA-regulated Transposable Elements}
}
\newline
% authors go here:
\\
%~ Arnaud Le~Rouzic\textsuperscript{1,*} \\
%~ Siddharth S.~Tomar\textsuperscript{1} \\
%~ Aur\'elie Hua-Van\textsuperscript{1}
%~ \\
\bigskip
\bf{1} Laboratoire Évolution: Génomes, Comportement, Écologie; Université Paris-Saclay, CNRS, IRD.
\\
\bigskip
* arnaud.le-rouzic@egce.cnrs-gif.fr

\end{flushleft}

\section*{Abstract}

\linenumbers

\section{Introduction}

Transposable elements (TEs) are repeated sequences that tend to accumulate in genomes, and often constitute a substantial part of eukaryotic DNA. In general, TE families are the most active upon their arrival in a new genome through a horizontal transfer; their copy number increases up to a maximum, at which point transposition stops. TE sequences are then progressively degraded and fragmented, accumulating sustitutions, insertions, and deletions, up to being undetectable and not identifiable as such. The reasons why the nature of TE families, the total TE content, and the number of copies per family vary substantially in the tree of life, even among close species, are far from being well-understood, which raises interesting challenges in comparative genomics. 

TEs spread in genomes by replicative transposition, which ensures both the genomic increase in copy number and the invasion of populations across generations of sexual reproduction \cite{CC83}. They are often cited as a typical example of selfish DNA sequences, as they can spread without bringing any selective advantage to the host species, and could even be deleterious \cite{OC80, DS80}. Even if an exponential amplification of a TE family could, in theory, lead to species extinction, empirical evidence rather suggests that TE invasion generally stops due to several (non-exclusive) physiological or evolutionary mechanisms, including selection, mutation, and regulation. Selection occurs whenever TE sequences are deleterious for the host species, as individuals carrying less TE copies will be favored by natural selection, and will thus reproduce preferentially, which tends to decrease the number of TE copies at the next generation. The mutation-control scenario relies on the degradation of the protein-coding sequence of TEs, which decreases the amount of functional transposition machinery (and thus the transposition rate) \cite{LCB07} \emph{(+ cite something about RIP?)}. Alternatively, substitutions or internal deletions in TEs could generate non-autonomous elements, able to use the transposition machinery without producing it, decreasing the transposition rate of autonomous copies \cite{LC06, RLZ+16}. Finally, transposition regulation refers to any mechanism involved in the control of the transposition rate of potentially functional TE sequences by the TE itself of by the host. 

\emph{Quick tour of known regulation mechanisms? Transcription silencing by methylation, alternative splicing (P element), OPI, etc.}. It has been recently confirmed that post-transcriptional TE silencing by small RNA was probably the most widespread and efficient TE regulation mechanism. \emph{Perhaps a quick presentation of the piwi pathway, + how widespread it is -- Drosophila, vertegrates, what about plants?}. 

If the strength of regulation increases with the copy number, the transposition rate is expected to drop in the course of the TE invasion up to the point where transposition stops. Nevertheless, the piwi regulation pathway displays unique features that may affect substantially the evolutionary dynamics of TE families: (i) it relies on a mutation-based mechanism, involving regulatory loci that may need several generations to appear (ii) the regulatory loci in the host genome segregate independently from the TE families and have their own evolutionary dynamics (the TE invades a genetically-heterogeneous populations, which are a mixture of permissive and repressive genetic backgrounds), and (iii) the regulation mechanism is irreversible (transposition does not occur again in low-copy number individuals). The consequences of these unique features on the TE invasion dynamics are not totally clear yet. Individual-based stochastic simulations have shown that (...  \cite{Kof19,KAZ18,CL10}). 


\section{General setting}

The model tracks two variables: (i) the average copy number, and (ii) the frequency of piRNA regulatory sites.

\paragraph{Symbols} \mbox{} \\

\begin{tabular}{cll}
Symbol & Meaning & Typical value \\ \hline
$N$ & Population size & $\infty$ \\
$\bar n$ & Average number of copies per individual & $1 > \bar n > 1000$ \\
$u_0$ & Maximal transposition rate & $u \leq 1$ \\
$v$ & Deletion rate & $v < 10^{-4}$ \\
$\pi$ & Proportion of new insertions in a piRNA region & $\pi \simeq 3\%$ \\
\end{tabular}

\paragraph {Model assumptions} The model assumes that genetic drif can be neglected. This implies not only that $N \rightarrow \infty$, but also that $\bar n \cdot N \rightarrow \infty$, which excludes the very beginning of a TE invasion after a horizontal transfer. 

The model also assumes linkage equilibrium in a sexual, random mating population. This only holds if transposition and selection are not too strong. It also assumes that the genome is freely recombining; predictions are probably inaccurate when sex chromosomes are large compared to the genome size. 

\paragraph {Historical models} The general setting dates back to Charlesworth \& Charlesworth 1983. 

\begin{equation}
\bar n_{t+1} = \bar n_t(u_n - v),
\end{equation}


If $u_n$ is constant, the dynamics is exponential, but if $u_0 > v$, $d u_n / d n < 0$, $\overline{u_n} \simeq u_{\bar n}$, and $\mathrm{lim} (u_n) < v$, with $u_{\hat n} = v$, then an equilibrium can be reached at $n = \hat n$. 

If the TEs are deleterious ($w_n < w_0$), then the effect of selection can be accounted for (the approximation being better when the fitness function $w_n$ is smooth). 

\begin{equation}\label{eq:cc2}
\bar n_{t+1} \simeq \bar n_t(u_n - v) + \bar n \frac{d \log w_n}{d n} \Bigr|_{\bar n}.
\end{equation}

Equilibrium is reached when $\bar n_{t+1} = \bar n_t = \hat n$, which can be computed when the fitness function $w_n$ is known. 

%~ \begin{figure}[h]
%~ \begin{center}
%~ << cc1, echo=FALSE, fig=TRUE, width=10, height=4 >>=
%~ source("amodel1.R")

%~ layout(t(1:2))
%~ plot(cc83(u=function(n) 0.4/(1+3*n), v=0.01, dlw=function(n) 0, Tmax=400), xlab="Generations", ylab="Copy number", type="l")
%~ plot(cc83(u=function(n) 0.1, dlw=function(n) -0.001*n, v=0.01, Tmax=50), xlab="Generations", ylab="Copy number", type="l")
%~ @
%~ \end{center}
%~ \caption{Different equilibria under the Charlesworth \& Charlesworth 1983 model: left, neutral model, the equilibrium is acheved when transposition regulation brings the transposition rate at the level of the deletion rate; right: selection model, the equilibrium is reached when selection is stronger than transposition. }
%~ \end{figure}

\section{piRNA model}

\subsection{Basic setting} 

The basic model setting assumes a single piRNA cluster in the genome, which segregates at frequency $p_t$ in an infinite population at generation $t$. The only mechanisms considered are replicative transposition and regulation. This first model assumes a haploid setting. The transposition rate per copy and per generation is $u$ in absence of regulation, and $0$ when a cluster is present in the genome. Assuming random mating and no linkage disequilibrium (i.e.\ no correlation between $n$ and the genotype at the regulatory cluster), when $\pi$ is the insertion probability in a pi-cluster,

\begin{align}\label{eq:basic}
\begin{split}
\Delta \bar n_t = n_{t+1} - n_t &= \bar n_t u (1-p_t) \\
\Delta p_t = p_{t+1} - p_t &= \pi \bar n_t u (1-p_t)
\end{split}
\end{align}

The system admits three equilibria: $E_1: u = 0$ (no transposition), $E_2: n_t = 0$ (loss of the transposable element), and $E_3: p_t = 1$ (fixation of the regulatory cluster). The dynamics of $n_t$ and $p_t$ are proportional ($\Delta p_t = \pi \Delta \bar n_t$, figure~\ref{fig:figA}), which means that under the reasonable assumption that $p_0 = 0$ (no regulatory clusters at the beginning of the invasion), $\bar n_t = n_0 + (1/\pi) p_t$. As a consequence, the full solution for the "cluster fixation" equilibrium $E_3$ is $p_\infty = 1, \bar n_\infty = n_0 + 1/\pi$. Strikingly, the copy number at equilibrium does not depend on the transposition rate. 

\begin{figure}[h]
\begin{adjustwidth}{-1in}{0in}
\begin{flushright}
	\includegraphics[width=1\textwidth]{../figures/figA}
\caption{\label{fig:figA} Dynamics in the basic piRNA model, $u=0.2, \pi=0.03$. Left: average number of copies, right: frequency of the segregating cluster in the population. }
\end{flushright}\end{adjustwidth}
\end{figure}


\begin{figure}[t]
\begin{adjustwidth}{-1in}{0in}
\begin{flushright}
\includegraphics[width=1\textwidth]{../figures/figB}
\caption{Influence of key parameters ($u$, $\pi$, and $n_0$) on the equilibrium state (number of copies at equilibrium and time to reach the equilibrium). }
\end{flushright}\end{adjustwidth}
\end{figure}


Time to equilibrium depends a lot on the transposition rate (large transposition rates reach equilibrium in a few generations, low transposition rates may need thousands). However, the number of copies at equilibrium does not depend on the transposition rate (!). The most important factor for the equilibrium number of copies is $\pi$, the relative size of pi-clusters in the genome. If clusters are rare, copies can accumulate fast before regulation kicks in. Finally, the equilibrium copy number is insensitive to $n_0$, but the time to regulation, slightly more. 


\subsection{Regulation dominance}

In diploid species, if the regulation is co-dominant (i.e., if the transposition rate is $u$ in genotypes deprived of clusters, $u/2$ in heterozygotes, and $0$ in homozygotes for the cluster), the dynamics of both cluster frequency and copy number is identical to the haploid case (equation~\ref{eq:basic}). However, due to the piRNA regulation mechanism, it is likely that regulation is dominant, i.e.\ a single cluster in heterozygotes is enough to ensure the total repression of transposition. In that case, the frequency of genotypes in which transposition is possible becomes $(1-p)^2$, and the dynamic model is:

\begin{align}\label{eq:dom}
\begin{split}
\Delta \bar n_t = \bar n_t u (1-p_t)^2 \\
\Delta p_t = \pi \bar n_t u (1-p_t)^2.
\end{split}
\end{align}

This only changes the rate at which the cluster invades the population, but not the equilibrium, which remains the same as in equation~\ref{eq:basic}. 

\subsection{Number of clusters}

2 clusters

n clusters

$\infty$ clusters

\subsection{Selection}

Natural selection, by favoring the reproduction of genotypes with fewer TE copies, generally acts in the same direction as regulation. A piRNA regulation model implementing selection could be derived by combining equations \ref{eq:cc2} and \ref{eq:basic}. In order to simplify the calculation, we will derive the results assuming that the deleterious effects of TE copies are independent, i.e.\ $w_n = \exp(- n s)$, where $n$ is the copy number and $s$ the coefficient of selection (deleterious effect per insertion), so that $\mathrm{d} \log w_n / \mathrm{d} n = -s$. 

\begin{equation}\label{eq:seln}
\Delta \bar n_t = \bar n_t u (1-p_t) - \bar n s,
\end{equation}

For the dynamics of cluster frequency, two variants can be defined, depending on whether the cluster copies are neutral or deleterious. If cluster TEs are neutral, the model remains:

\begin{equation}\label{eq:selpn}
\Delta p_t = \pi \bar n_t u (1-p_t).
\end{equation}

The model obtained by combining equations~\ref{eq:seln} and~\ref{eq:selpn} only gives two equilibria, $E_2: \bar n = 0$, and $E_3: s=0$ and $p=1$, which is the same as for equation~\ref{eq:basic} (no selection and fixation of the regulatory cluster). At the beginning of the dynamics, assuming $p_0 = 0$, the TE invades if $u > s$ (otherwise the system converges to equilibrium $E_2$ and the TE is lost). The copy number increases up to a maximum $n^\ast$, where $p^\ast = 1 - s/u$ and $n^\ast = n_0 + 1/\pi - \log (1-s/u) \simeq n_0 1/\pi + s/u$ if $s \ll u$. 

\noindent but if the cluster insertions are as deleterious as other TEs, the model becomes:

\begin{equation}\label{eq:selps}
\Delta p_t = \pi \bar n_t u (1-p_t) - s p_t \frac{1-p_t}{1- s p_t}.
\end{equation}


\begin{table}
\begin{adjustwidth}{-2in}{0in}
\begin{flushright}
\begin{tabular}{lp{4cm}p{4cm}p{4cm}}
 & Neutral & Deleterious TEs & Deleterious TEs and clusters \\ \hline
1 codominant cluster 		& \begin{equation*}\begin{cases} \hat n = n_0 + 1/\pi \\ \hat p = 1 \end{cases}\end{equation*} 
							& \begin{equation*}\begin{cases} n^\ast = n_0 + \frac{1}{\pi}(1+ \frac{s}{u}\log \frac{s}{u}) \\ p^\ast = 1 - s/u \end{cases}\end{equation*}
							\begin{equation*}\begin{cases} \hat n = 0 \\ \hat p = 1 - \exp[-\frac{u\pi}{s}(n_0+\frac{1}{\pi})] \end{cases}\end{equation*} 
							& \begin{equation*}\begin{cases} \hat n =\frac{1}{u \pi} \frac{1}{s(1-s/u)-1} \\ \hat p = \frac{1}{s^2(1-s/u)} \end{cases}\end{equation*}  \\
1 dominant cluster 			& \begin{equation*}\begin{cases} \hat n = n_0 + 1/\pi \\ \hat p = 1 \end{cases}\end{equation*} 
							& \begin{equation*}\begin{cases} n^\ast = n_0 + \frac{1}{\pi}[1+ \frac{s}{u}(1-\sqrt{\frac{u}{s}})] \\ p^\ast = 1 - \sqrt{\frac{s}{u}} \end{cases}\end{equation*}
							\begin{equation*}\begin{cases} \hat n = 0 \\ \hat p = 1-\frac{1}{1-u(n_0 \pi + 1)/s} \end{cases}\end{equation*} 
							& \begin{equation*}\begin{cases} \hat n =\frac{1}{u \pi} \frac{1}{s(1-\sqrt{s/u})-1} \\ \hat p \simeq \frac{1 - \sqrt{1-4s}}{2s} \end{cases}\end{equation*} \\
$n$ dominant clusters & & & \\
$\infty$ dominant clusters & & & \\
\end{tabular}
\end{flushright}\end{adjustwidth}
\end{table}



\printbibliography


\end{document}
