\documentclass[10pt,a4paper]{article}
\usepackage[utf8]{inputenc}


%~ \usepackage{cite}
%~ \makeatletter
%~ \renewcommand{\@biblabel}[1]{\quad#1.}
%~ \makeatother


\usepackage{nameref}
\usepackage[colorlinks=true,citecolor={Gray}]{hyperref}
\usepackage[right]{lineno}

\usepackage{url}

\usepackage[title]{appendix}
%~ \renewcommand{\appendixname}{Appendix}

\usepackage[top=0.85in,left=2.75in,footskip=0.75in,marginparwidth=2in]{geometry}

% headrule, footrule and page numbers
\usepackage{lastpage,fancyhdr,graphicx}
\usepackage{epstopdf}
\pagestyle{myheadings}
\pagestyle{fancy}
\fancyhf{}
\rfoot{\thepage/\pageref{LastPage}}
\renewcommand{\footrule}{\hrule height 2pt \vspace{2mm}}
\fancyheadoffset[L]{2.25in}
\fancyfootoffset[L]{2.25in}

\raggedright
\setlength{\parindent}{0.5cm}
\textwidth 5.0in 
\textheight 8.75in

\usepackage{changepage}

\usepackage{microtype}
\DisableLigatures[f]{encoding = *, family = * }

\usepackage{color}
\definecolor{Gray}{gray}{.25}

\usepackage{array}
\newcolumntype{L}[1]{>{\raggedright\let\newline\\\arraybackslash\hspace{0pt}}m{#1}}

\usepackage{graphicx}
\usepackage{sidecap}
\usepackage{wrapfig}
\usepackage[pscoord]{eso-pic}
\usepackage[fulladjust]{marginnote}
\reversemarginpar

\usepackage[natbib=true, style=authoryear, uniquename=false, maxcitenames=2, maxbibnames=99, firstinits=true, doi=false, url=false]{biblatex}
\bibliography{amodel.bib}
\renewbibmacro{in:}{}                                  % remove In:
\DeclareFieldFormat[article, inbook]{title}{#1}        % remove quotes around titles
\AtEveryBibitem{%                                      % remove language indications
  \clearlist{language}%
}

\usepackage{bm}
\usepackage{amsmath}
\usepackage{mathtools}

\usepackage[english]{babel}



\title{}
\date{}

\begin{document}
\vspace*{0.35in}

% title goes here:
\begin{flushleft}
{\Large
\textbf\newline{A Population Genetics Theory for piRNA-regulated Transposable Elements}
}
\newline
% authors go here:
\\
%~ Arnaud Le~Rouzic\textsuperscript{1,*} \\
%~ Siddharth S.~Tomar\textsuperscript{1} \\
%~ Aur\'elie Hua-Van\textsuperscript{1}
%~ \\
\bigskip
\bf{1} Laboratoire Évolution: Génomes, Comportement, Écologie; Université Paris-Saclay, CNRS, IRD.
\\
\bigskip
* arnaud.le-rouzic@egce.cnrs-gif.fr

\end{flushleft}

\section*{Abstract}

We provide analytical expressions for maximum and equilibrium copy numbers and cluster frequencies, and we characterized 


\linenumbers

\section{Introduction}

Transposable elements (TEs) are repeated sequences that tend to accumulate in genomes, and often constitute a substantial part of eukaryotic DNA. According to the consensual "TE life cycle" model \citep{KL01}, TE families are the most active upon their arrival in a new genome; their copy number increases up to a maximum, at which point transposition slows down. TE sequences are then progressively degraded and fragmented, accumulating sustitutions, insertions, and deletions, up to being undetectable and not identifiable as such. The reasons why the nature of TE families, the total TE content, and the number of copies per family vary substantially in the tree of life, even among close species, are far from being well-understood, which raises interesting challenges in comparative genomics. 

TEs spread in genomes by replicative transposition, which ensures both the genomic increase in copy number and the invasion of populations across generations of sexual reproduction. They are often cited as a typical example of selfish DNA sequences, as they can spread without bringing any selective advantage to the host species, and could even be deleterious \cite{OC80, DS80}. Even if an exponential amplification of a TE family could, in theory, lead to species extinction, empirical evidence rather suggests that TE invasion generally stops due to several (non-exclusive) physiological or evolutionary mechanisms, including selection, mutation, and regulation. Selection limits the TE spread whenever TE sequences are deleterious for the host species: individuals carrying less TE copies will be favored by natural selection, and will thus reproduce preferentially, which tends to decrease the number of TE copies at the next generation (\cite{CC83}). The mutation-control scenario relies on the degradation of the protein-coding sequence of TEs, which decreases the amount of functional transposition machinery (and thus the transposition rate) \cite{LC06} \emph{(+ cite something about RIP?)}. Alternatively, substitutions or internal deletions in TEs could generate non-autonomous elements, able to use the transposition machinery without producing it, decreasing the transposition rate of autonomous copies \cite{LC06, RLZ+16}. 

Transposition regulation refers to any mechanism involved in the control of the transposition rate of potentially functional TE sequences by the TE itself of by the host. There is a wide diversity of known transposition regulation mechanisms; some prevent the transcription of the TE genes, others target the  transcript or the protein, etc. 

\emph{Quick tour of known regulation mechanisms? Transcription silencing by methylation, alternative splicing (P element), OPI, etc.}. It has been recently confirmed that post-transcriptional TE silencing by small RNA was probably the most widespread and efficient TE regulation mechanism. \emph{Perhaps a quick presentation of the piwi pathway, + how widespread it is -- Drosophila, vertegrates, what about plants?}. 

If the strength of regulation increases with the copy number, the transposition rate is expected to drop in the course of the TE invasion up to the point where transposition stops. Nevertheless, the piwi regulation pathway displays unique features that may affect substantially the evolutionary dynamics of TE families: (i) it relies on a mutation-based mechanism, involving regulatory loci that may need several generations to appear (ii) the regulatory loci in the host genome segregate independently from the TE families and have their own evolutionary dynamics (the TE invades a genetically-heterogeneous populations, which are a mixture of permissive and repressive genetic backgrounds), and (iii) the regulation mechanism is irreversible (transposition does not occur again in low-copy number individuals). The consequences of these unique features on the TE invasion dynamics are not totally clear yet. Individual-based stochastic simulations have shown that (...  \cite{Kof19,KAZ18,CL10}). 


\section{Methods}

\subsection{Historical framework}

Model setting and notation traces back to \cite{CC83}, who proposed to track the average TE copy number $\bar n$ in a population through the difference equation:

\begin{equation}
\bar n_{t+1} = \bar n_t + \bar n_t(u - v),
\end{equation}

\noindent where $u$ is the transposition rate (more exactly, the amplification rate per copy and per generation), and $v$ the deletion rate. 
In this neutral model, if $u$ is constant, the dynamics is exponential. If the transposition rate $u_n$ is regulated by the copy number ($u_0 > v$, $d u_n / d n < 0$), $\overline{u_n} \simeq u_{\bar n}$, and $\mathrm{lim} (u_n) < v$, with $u_{\hat n} = v$, then a stable equilibrium copy number $\hat n$ can be reached. 

However, in most organisms, TEs are probably not neutral. If TEs are deleterious, fitness $w$ decreases with the copy number ($w_n < w_0$). As a consequence, individuals carrying more copies reproduces less, which decreases the average copy number every generation. The effect of selection can be accounted for using traditional quantitative genetics, considering the number of copies $n$ as a quantitative trait: $\Delta \bar n \simeq \mathrm{Var}(n) \mathrm d \log(w_n)/ \mathrm d n$, where $\mathrm{Var}(n)$ is the variance in copy number in the population, and $\mathrm d \log(w_n)/ \mathrm d n$  approximates the selection gradient on $n$. The approximation is better when the fitness function $w_n$ is smooth and the copy number $n$ is not close to 0. Assuming random mating and no linkage disequilibrium, $n$ is approximately Poisson-distributed in the population, and $\mathrm{Var}(n) \simeq \bar n$. 

\cite{CC83} propose to add up the effects of transposition and selection to approximate the variation in copy number across generations:

\begin{equation}\label{eq:cc2}
\bar n_{t+1} \simeq \bar n_t(u_n - v) + \bar n \frac{d \log w_n}{d n} \Bigr|_{\bar n}.
\end{equation}

Equilibrium is reached when $\bar n_{t+1} = \bar n_t = \hat n$, which can be computed when the fitness function $w_n$ is known. 

\subsection{Numerical methods}

Model dynamics and data analysis was performed with R version 4.0 \citep{R20}. Model analysis was performed with packages deSolve \citep{SPS10}. 


\section{Results}

\subsection{Neutral trap model} 

The model assumes $k$ identical piRNA clusters in the genome, and the total probability to transpose in a cluster region is $\pi$. Each cluster locus can harbor two alleles: a regulatory allele (i.e., the cluster carries a TE insertion), which segregates at frequency $p$, and an "empty" allele (frequency $1-p$). Allele frequencies at all clusters were considered to be the same (a consequence of the infinite population assumption). If the regulatory allele frequency at generation $t$ is $p_t$, the average number of cluster insertions for a diploid individual is $2kp_t$. Deletions were neglected ($v=0$). The presence of a single regulatory allele at any cluster was supposed to trigger complete regulation: the transposition rate per copy and per generation was $u$ in absence of regulation, and $0$ otherwise. Assuming random mating and no linkage disequilibrium (i.e.\ no correlation between $n$ and the genotype at the regulatory cluster),

\begin{align}\label{eq:basic}
\begin{split}
\Delta \bar n_t = n_{t+1} - n_t &= \bar n_t u (1-p_t)^{2k} \\
\Delta p_t = p_{t+1} - p_t &= \frac{\pi}{k}  \bar n_t u (1-p_t)^{2k}.
\end{split}
\end{align}

We approximated this discrete (non-overlapping) generation model with a continuous process, and the neutral model was rewritten as a set of two differential equations on $\bar n$ (relabelled $n$ for simplicity) and $p$:

\begin{align}\label{eq:neutral}
\begin{split}
\frac{\mathrm d n}{\mathrm d t} &= n u (1-p)^{2k} \\
\frac{\mathrm d p}{\mathrm d t} &= \frac{\pi}{k} n u (1-p)^{2k}.
\end{split}
\end{align}

Initial conditions were fixed to $n_0 = 1$ and $p_0 = 0$. 

The system admits three equilibria: $E_1: u = 0$ (no transposition), $E_2: \hat n = 0$ (loss of the transposable element), and $E_3: \hat p = 1$ (fixation of the regulatory cluster). Equilibria $E_1$ and $E_2$ do not deserve to be investigated further, as $u=0$, $n_0=0$, or $p_0=1$ do trivially result in the absence of any TE invasion. Equilibrium $E_3$ is analytically tractable, as $\mathrm d n / \mathrm d p = k/\pi$, and $n = n_0 + pk/\pi$ at any point of time. 

\begin{equation}\label{eq:eqneutral}
\begin{cases}
\hat n &= n_0 + k/\pi \\
\hat p &= 1.
\end{cases}
\end{equation}

Cluster fixation is asymptotic ($\lim_{t \rightarrow \infty} p = 1$), and the equilibrium is asymptotically stable ($\mathrm d n /\mathrm d t > 0$ and $\mathrm d p/ \mathrm d t > 0$). 

\begin{figure}[h]
\begin{adjustwidth}{-1in}{0in}
\begin{flushright}
	\includegraphics[width=1\textwidth]{../figures/figA}
\caption{\label{fig:figA} Dynamics in the neutral piRNA model, $n_0=1, \pi=0.03$. The top panel illustrates the influence of the transposition rate, the bottom panel of the number of clusters. Left: number of copies $n$, right: frequency of the segregating clusters in the population ($p$). Open symbols: simulations, plain lines: difference equations, hyphenated lines: predicted equilibria. }
\end{flushright}\end{adjustwidth}
\end{figure}



\subsection{Selection}

Natural selection, by favoring the reproduction of genotypes with fewer TE copies, generally acts in the same direction as regulation. A piRNA regulation model implementing selection could be derived by combining equations \ref{eq:cc2} and \ref{eq:basic}. In order to simplify the analysis, we derived the results assuming that the deleterious effects of TE copies were independent, i.e.\ $w_n = \exp(- n s)$, where $n$ is the copy number and $s$ the coefficient of selection (deleterious effect per insertion), so that $\mathrm{d} \log w_n / \mathrm{d} n = -s$. 

Numerical simulations suggested that the approximation suggested by \citep{CC83} (equation~\ref{eq:cc2}), based on the independence of transposition and selection, was rather unconvincing when the transposition rate was high, typically, at the beginning of the dynamics. This discrepancy is mainly due to the fact that transposition overdisperses the copy number distribution compared to the theoretical Poisson: after transposition, the copy number rises to $\bar n = \bar n(1+u)$, while its variance becomes $\mathrm V(n) = \bar n (1+u)^2$. Neglecting terms in $u^2$, the apparent efficiency of selection $s^\prime$ thus seem to increase by a factor $1+2u$. Note that this correction is subtle and depends on model details (transposition before or after reproduction, immediate or delayed fitness penalty of new TE insertions). The correction $s^\prime \simeq s(1+2u)$ was used here solely to match the arbitrary modelling options of the simulation routine. 

The following relies on the additional assumption that $1-\pi \simeq 1$, (leading to $n \gg 2kp$, i.e.\ that the number of TE copies in the clusters is never large enough to make a difference in the total TE count). We will describe two selection scenarios that happened to lead to qualitatively different outcome: (i) TE insertions in pi-clusters are neutral, and (ii) TE insertions in pi-clusters are as deleterious as the other insertions. 

\paragraph{Deleterious TEs \& neutral clusters} If cluster TEs are neutral, the model becomes:

\begin{align}\label{eq:dtnc}
\begin{split}
\frac{\mathrm d n}{\mathrm d t} &= n u (1-p)^{2k} - n s^\prime \\
\frac{\mathrm d p}{\mathrm d t} &= \frac{\pi}{k} n u (1-p)^{2k}.
\end{split}
\end{align}

This equation only gives two equilibria, $E_2: \bar n = 0$, and $E_3: s^\prime=0$ and $p=1$, which is the same as for the neutral model (equation~\ref{eq:neutral}): no selection and fixation of all regulatory clusters. At the beginning of the dynamics, assuming $p_0 = 0$, the TE invades if $u > s^\prime$ (otherwise the system converges immediately to equilibrium $E_2$ and the TE is lost). The copy number increases ($\mathrm d n/\mathrm d t > 0$) up to a maximum $n^\ast$, which is achieved when $p=p^\ast$. The maximum copy number can be obtained analytically (Appendix~\ref{app:maxdtnc}):

\begin{equation}\label{eq:maxdtnc}
\begin{split}
	p^\ast &= 1-\left(\frac{s^\prime}{u}\right)^\frac{1}{2k} \\
	n^\ast &= n_0 + \frac{k}{\pi}\left[ 1- \frac{1}{2k-1} \left( 2k \left( \frac{s^\prime}{u} \right)^\frac{1}{2k} - \frac{s^\prime}{u} \right) \right].
\end{split}  
\end{equation}

Once the maximum number of copies is achieved, cluster copies keep on accumulating, decreasing the transposition rate, which leads to a decrease in the copy number, up to the loss of the the element ($\hat n = 0$ at equilibrium). At that stage, clusters are not fixed, and the equilibrium cluster frequency $\hat p$ can be expressed as a function of copy number and cluster frequency at the maximum ($p^\ast$ and $n^\ast$) (Appendix~\ref{app:exhatp}:

\begin{equation}\label{eq:exhatp}
	\hat{p} - \frac{s^\prime}{u(2k-1)}\frac{1}{(1-\hat p)^{2k-1}} = p^\ast - \frac{s^\prime}{u(2k-1)}\frac{1}{(1-p^\ast)^{2k-1}} - \frac{\pi n^\ast}{k},
\end{equation}

\noindent from which an exact solution for $\hat p$ cannot be calculated. The approximation:
\begin{equation}\label{eq:approxhatp}
	\hat p \simeq 1-\left[\frac{u}{s^\prime}(2k-1)p^\ast +1\right]^\frac{1}{1-2k}
\end{equation}

\noindent could be derived (Appendix~\ref{app:approxhatp}, and happened to be acceptable for a wide range of transposition rates and for small selection coefficients ($s < 0.1$) (Figure~\ref{fig:figC2}). 

\begin{figure}[h]
\begin{adjustwidth}{-1in}{0in}
\begin{flushright}
	\includegraphics[width=1\textwidth]{../figures/figC}
\caption{\label{fig:figC} Dynamics of the deleterious TE - neutral cluster model. The top panel illustrates the influence of the selection coefficient $s$, the bottom panel of the number of clusters $k$. Left: number of copies $n$, right: frequency of the segregating clusters in the population ($p$). Open symbols: simulations, plain lines: difference equations, hyphenated lines: predicted equilibria. }
\end{flushright}\end{adjustwidth}
\end{figure}

\begin{figure}[h]
\begin{adjustwidth}{-1in}{0in}
\begin{flushright}
	\includegraphics[width=1\textwidth]{../figures/figC2}
\caption{\label{fig:figC2} Equilibrium cluster frequency $\hat p$ for the deleterious TE - neutral cluster model as a function of the transposition rate $u$ and the selection coefficient $s$. The number of clusters $k$ is indicated with different line types. The approximation proposed in equation \ref{eq:approxhatp} is illustrated in gray. }
\end{flushright}\end{adjustwidth}
\end{figure}

Equation~\ref{eq:maxdtnc} can be reorganized to address the problem of population exinction, as formulated in \citet{Kof20}. Even if the final equilibrium state involves the loss of all TE copies, populations need to go through a stage where up to $n^\ast$ deleterious copies are present in the genome. Accordingly, an arbitrary fitness threshold $w_c$ can be set, below which the population is consider to risk extinction.  This makes it possible to derive the critical cluster size $\pi_c$ to avoid population extinction:

\begin{equation} \label{eq:critsize}
\pi_c > \frac{k}{-(\log w_c)/s - n_0} \left[ 1- \frac{1}{2k-1} \left( 2k \left( \frac{s^\prime}{u} \right)^\frac{1}{2k} - \frac{s^\prime}{u} \right) \right]. 
\end{equation}

Setting $s=0.01$, $u=0.1$, and $n_0 = 1$, as in the other examples, and taking $w_c = 0.1$ gives $\pi_c > 0.0018$ for $k=1$ and $\pi_c > 0.0025$ for $k=5$, these values being very similar to the interval $0.1\%$ to $0.2\%$ determined numerically by \citet{Kof20}. 

\paragraph{Deleterious TEs and deleterious clusters} If the cluster insertions are as deleterious as other TEs, selection acts on cluster frequency as predicted by population genetics (assuming no dominance): 
\begin{equation}\label{eq:selps}
\begin{split}
\frac{\mathrm d n}{\mathrm d t} &= n u (1-p)^{2k} - n s^\prime \\
\frac{\mathrm d p}{\mathrm d t} &= \frac{\pi}{k} n u (1-p)^{2k} - s p \frac{1-p}{1- s p}.
\end{split}
\end{equation}

This allows for a new equilibrium $E_4$:

\begin{equation}\label{eq:selpseq}
\begin{cases}
\displaystyle \hat n = \frac{k}{\pi}\frac{s}{s^\prime} \left(\frac{s^\prime}{u}\right)^\frac{1}{2k}\frac{\hat p}{1-s\hat p} \\
\displaystyle \hat p = 1 - \left(\frac{s^\prime}{u}\right)^\frac{1}{2k}.
\end{cases}
\end{equation}

The equilibrium exists ($\hat n > 0$ and $\hat p > 0$) whenever $s < u(1+2u)$, i.e.\ the transposition rate must be substantially larger than the selection coefficient. 

\begin{figure}
\begin{adjustwidth}{-2in}{0in}
\begin{center}
	\includegraphics[width=15cm]{../figures/figD}
\caption{\label{fig:figD} Dynamics of the copy number ($n$, left) and the cluster frequency ($p$, right) in the deleterious TE - deleterious cluster model. Top panels: influence of the selection coeffcient, bottom panels: influence of the number of clusters. Plain lines: predicted dynamics from the differential equations (eq.~\ref{eq:selps}), hyphenated lines: predicted equilibrium (eq.~\ref{eq:selspeq}), dots: simulations. }
\end{center}\end{adjustwidth}
\end{figure}


\begin{figure}
\begin{adjustwidth}{-1in}{0in}
\begin{center}
	\includegraphics[width=15cm]{../figures/figE}
\caption{\label{fig:figE} Equilibrium copy number ($\hat n$, black) and cluster frequency ($\hat p$, red) as a function of model parameters ($u$, $s$, and $\pi$) in the deleterious TEs - deleterious cluster model. Default parameter values were $u=0.1$, $s=0.01$, and $\pi=0.03$. The number of clusters ($k=1$, $k=2$, and $k=5$) is indicated by different line styles. }
\end{center}\end{adjustwidth}
\end{figure}

A linear stability analysis (Appendix~\ref{app:selpseq}) shows that for the whole range of $u$, $\pi$, and $k$, as well as for most of the reasonable values of $s$, the equilibrium is a stable focus, i.e.\ the system converges to the equilibrium while oscillating around it. 

\begin{figure}
\begin{adjustwidth}{-1in}{0in}
\begin{center}
	\includegraphics[width=5cm]{../figures/figG}
\caption{\label{fig:figG} Equilibrium stability for the deleterious TES - deleterious cluster model. The figure represents the real part of the first eigenvalue of the Jacobian matrix for two major parameters ($u$ and $s$), with $k=1$ and $\pi = 0.03$. The eigenvalue is negative for the whole parameter range, and is a complex number for most of the range (below the red line). The purple line delineates $s = u/(1+2u)$, beyond which selection is too strong to let the TE invade (white area). }
\end{center}\end{adjustwidth}
\end{figure}

\subsection{Genetic drift}

The models described above assume infinite population sizes, which may not hold for low-census species and for laboratory (experimental evolution) populations. We assessed the influence of population size on the copy number with numerical simulations, comparing the neutral model, the deleterious TE - neutral cluster model, and the deleterious TE - deleterious cluster model with a "classical" copy-number regulation model in which $u_n = u_0/(1+kn)$. Since the deleterious TE - neutral cluster model does not allow for an equilibrium, comparisons had to be performed before the stabilization of the copy number (arbitrarily, at T=100 generations). Models were parameterized such that the copy number $n$ was approximately the same after 100 generations. Drift affects piRNA models substantially more than copy number regulation, the variance of all trap models being approximately one order of magnitude larger (Figure~\ref{fig:figH2}). The neutral trap model appears to be slightly more sensitive to drift. Consistently with population genetics theory, the variance across simulation replicates decreased with $1/N$ for all models. 

\begin{figure}
\begin{adjustwidth}{0in}{0in}
\begin{center}
	\includegraphics[width=5cm]{../figures/figH3}
	\includegraphics[width=5cm]{../figures/figH2}
\caption{\label{fig:figH2} A: The effect of genetic drift is larger in the trap model than for copy-nuber regulation models. The figure displays the average copy number $\bar n$ in 20 independent replicates, $N=100$ for both models. Parameters were $u=0.1$, $s=0.01$, $\pi=0.03$, $k=2$ for the trap model, and $u_n=0.1/(1+0.44n)$, and $s=0.01$, for the copy-number regulation model. Regulation strengh was set so that the expected equilibrium copy number $\hat n \simeq 13.6$ was the same for both models. B: Variance in the average copy number (relative to the average copy number) at generation 100 among replicated simulations for various population sizes. Four models are displayed: neutral trap model, Deleterious TE - neutral clusters, Deleterious TE - deleterious cluster, and copy number regulation. Models were parameterized so that they have very similar copy numbers (about 18) at generation 100; Neutral trap model: $u=0.045$, $\pi=0.03$, $k=2$); Deleterious TE - neutral clusters: $u=0.13$, $\pi=0.03$, $s=0.01$, $k=2$; Deleterious TE - deleterious clusters: $u=0.07$, $pi=0.03$, $s=0.01$, $k=2$; Copy number regulation: $u_n = 0.17/(1+0.44n)$, $s=0.01$. The theoretically-expected decrease in variance (in $1/N$) is illustrated for the neutral piRNA model (slope of $-1$ on the log-log plot). }
\end{center}\end{adjustwidth}
\end{figure}

The standard population genetics theory predicts that selection is less effective at eliminating slightly deleterious alleles. Assuming that TE copies are deleterious, they should be eliminated faster in large populations compared to small ones. Although this mechanism has been proposed to explain the accumulation of junk DNA (including TE copies) in multicellular eukaryotes \citep{LC03}, little is known about how the equlibrium copy number of an active TE family is expected to be affected by drift even in the simplest scenarios \citep{CC83}. Yet, informal models suggest that drift may have a limited effect, as copy number equilibria rely on the assumption that evolutionary forces that limit TE amplification (regulation and/or selection) increase in intensity when the copy number increases. Thus, when drift pushes the average copy number up or down, TE amplification is expected to be less or more effective respectively, which compensates the random deviation. Simulations show that, whatever the model, the copy number is indeed slightly higher in small populations ($N < 100$) for , but this effect never exceeds 20\% of the total copy number (Figure~\ref{fig:figH1}). Overall, drift has a very limited impact on the copy number. 

\begin{figure}
\begin{adjustwidth}{-1in}{0in}
\begin{center}
	\includegraphics[width=15cm]{../figures/figH1}
\caption{\label{fig:figH1} Distribution of the average copy number $\bar n$ among 1000 replicates in different models, with various population sizes. Models were parameterized so that they achieve very similar average copy numbers ($\bar n \sim 19$) in large populations (horizontal dotted line): $s=0.01$ for all models (except the neutral model), $k=1$ cluster and $\pi =0.03$ in all trap models. Transposition rates were: $u=0.045$ for the neutral model, $u=0.05$ for the Deleterious TE - neutral cluster model, $u=0.15$ for the Deleterious TE - Deleterious cluster model, and $u_n=0.17/(1+0.45 n)$ for the regulation model. }
\end{center}\end{adjustwidth}
\end{figure}


\section{Discussion}

The formalization of TE regulation by pi-RNA clusters (the "trap model") made it possible to derive a series of non-intuitive results, and evidence how trap regulation differs from traditional (copy-number based) regulation models. Among the most striking results: (i) in absence of selection (neutral trap model), the equilibrium copy number does not depend on the transposition rate, (ii) the efficiency of regulation increases with the size of clusters and decreases with the number of clusters, (iii) deleterious TEs can always invade when the transposition rate is larger than the selection coefficient, but the TE family can persist only if copies inserted in clusters are deleterious as well (otherwise, clusters increase in frequency up to the loss of all non-cluster TE copies). Equilibria are always stable. Pi-RNA regulation being a mutational process, the TE copy number is substiantially more sensitive to genetic drift than other regulation models. 

These results confirm and formalizes previous work based on numerical simulations, in particular from \cite{Kof19} who has already pointed out the small effect of transposition rate on the final state of the population and the inverse relationship between the number of clusters and the number of TE copies. The characterization of the equilibria demonstrate how the neutral trap model differs from the transposition-selection balance model proposed by \cite{CC83}; while the transposition-selection balance mostly depends on the transposition rate, the trap model equilibrium is determined by the mutational target (the size and the number of pi-clusters). 

While the equilibrium for the neutral trap model can be expressed with a very simple formula (equation~\ref{eq:neutral}), the derivation of copy number and cluster frequencies is less straightforward when selection is accounted for (equations~\ref{eq:selps} and~\ref{eq:selpsseq}). In all cases, the TE copy number increases with the number of clusters, and decreases with the total cluster size. When TEs are deleterious even when inserted in the clusters, the equilibrium copy number depends on the transposition rate $u$ and the selection coeffcient $s$ in a non-monotonous way (less copies when $u$ or $s$ are either very low or very large). The fact that there exists an optimal transposition rate when TE insertions are deleterious have been proposed previously \citep{LC05}, but the optimal rate in the trap model (about 0.1 to 0.2 transpositions per copy and per generation in unregutaed genetic backgrounds, figure~\ref{fig:figE} seems more compatible with empirical estimates \citep{RLZ+16}. 

\subsection{Model approximations}

The mathematical formulation of the trap model relies on a series of approximations. The general framework is strongly inspired from \cite{CC83}, and is based on the same assuptions, such as a uniform transposition rates and selection coeffcients among TE copies, diploid, random mating populations, and no linkage disequilibrium. This framework fits better some model species, including Drosophila or humans, than others (plants, nematodes...) for which the population genetics setup needs to be adapted. In order to improve the mathematical tractability of the model, we turned difference equations (corresponding to non-overlapping generations) to differential equations (continuous time), which happend to have a minor effect on the results given the convincing fit of individual-based (non-overlapping generations) simulations. This approximation may, however, not hold in extreme conditions (especially with very high transposition rates or selection coeffcients). 

The biology of the pi-cluster regulation was also simplified. We considered that pi-clusters were completely dominant and epistatic, i.e.\ a single insertion drives the transposition rate to zero. Relaxing slightly this assumption is unlikely to modify qualitatively the model output, e.g.\ considering that regulatory insertions are recessive  would change the frequency of permissive genotypes from $(1-p)^{2k}$ to $(1-p^2)^k$, which would increase the cluster frequency at equilibrium but not its stability. In contrast, imperfect regulation (a residual transposition rate even when clusters are fixed, such as in \cite{LC10}) would break the equilibrium in the neutral case, and copy number would raise exponentially. When TEs are deleterious, a residual transposition effect would have a much smaller effect though. 

In order to compute the equilibria, we assumed no epistasis on fitness, i.e.\ constant $\mathrm d w / \mathrm d n = s$. Deriving the model with a different fitness function is possible, although solving the differential equations could be more challenging. Instead of our fitness function $w_n = e^{-ns}$, \cite{CC83} proposed $w_n = 1 - sn^c$ ($c$ being a coeffcient quantifying the amount of epistasis on fitness), while \cite{DC06} later used $w_n = e^{-sn - cn^2}$. Considering negative epistasis on fitness (i.e.\ the cost of additional deleterious mutations increases) in TE population genetic models has two major interests: (i) the strength of selection increasing with the copy number, it ensures and stabilizes the equilibrium even in absence of regulation (\cite{CC83}), and (ii) the model is more realistic, as epistasis on fitness has been measured repeatedly on many organisms \cite{MRF+05,KSB07,KDS+11}. Yet, for the trap model, regulation itself is strong enough to achieve an equilibrium in absence of selection, so epistasis on fitness is expected to modify the equilibrium copy number and the range of parameters for which a reasonable copy number can be maintained (\cite{Kof19}), but not the presence of a theoretical equilibrium. 

\subsection{pi-RNA clusters and recombination}

\cite{Kof19} has already noticed that recombination reduces cluster efficiency. For a given proportion of the genome $\pi$ occupied by piRNA clusters, regulation is more efficient with one large, non-recombining cluster (the "flamenco" model, according to the terminology from \cite{Kof19}) than with many small clusters spread on several chromosomes. Indeed, when several clusters segregate, recombination decreases the heritability of the transposition regulation (recombination between clusters can generate permissive genotypes in the offspring of a cross between transposition resistant individuals). In a similar way, regulation efficiency is expected to decrease with the within-cluster recombination rate (not modeled here). We confirmed here that the number of copies at equilibrium is indeed expected to be proportional to the number of clusters $k$. Selection for TE regulation should thus minimize recombination within and across clusters; the fact that, in most organisms, pi-clusters seem to be located at several loci needs to be explained by other factors than the regulation efficiency. 

An interesting hypothesis was raised by \cite{Kel18} about the possibility that pi-cluster frequency could be influenced by positive selection. Assuming deleterious TEs, genotypes able to control TE spread are indeed expected to display a selective advantage over those in which transposition is unregulated, suggesting that regulatory clusters should sweep in the population as advantageous alleles. Our model, neglecting linkage disequilibrium between TEs and clusters, would then underestimate the increase in frequency of regulatory alleles (and thus overestimate the copy number). Although the reasoning is theoretically valid, the actual strength of positive selection on pi-clusters depends on details about the selection mechanism. Assuming that TE insertions have a local deleterious effect (because they disrupt genes or gene regulation), the selective advantage is weak and indirect (of the same order of magnitude as $n s u$, the deleterious effect of the few insertions arising in a single generation), and is unlikely to alter the TE invasion dynamics (as shown by the good fit of our model vs.\ individual-based simulations). In contrast, if active transposition is deleterious (such as in the hybrid dysgenesis scenario explored by \cite{Kel18}), the selective advantage of regulatory cluster alleles is of the order of magnitude of $n s$ (at least when few clusters are segregating), and selection may have an effect on cluster frequencies. Although it is experimentally difficult to determine how selection acts on TEs, both scenarios are expected to leave different genomic footprints, as the positive selection hypothesis posits that regulatory alleles should be shared among many individuals of the population, while the non-selection hypothesis expects that various individuals are regulated by independent cluster insertions. Empirical evidence is scarce, but seems to favor the mutational hypothesis \cite{ZPK20}. 

\subsection{Concluding remarks}

Importance of mdoels to understand and interpret genomic TE patterns. 

Extension of the existing theory, new family of equilibrium. 


\printbibliography


\newpage 

\begin{appendices}
\renewcommand{\thefigure}{A\arabic{figure}}
\setcounter{figure}{0}

\section{Mathematical details}

\subsection{\label{app:maxdtnc} Equation \ref{eq:maxdtnc}}

When the copy number $n$ achieves its maximum $n^\ast$, $\mathrm d n/\mathrm d t = 0$. This happens when the cluster frequency $p^\ast$ is:

\begin{equation*}
\begin{split}
	\frac{\mathrm d n}{\mathrm d t} &= n^\ast u (1-p^\ast)^{2k} - n^\ast s^\prime = 0 \\
	p^\ast &= 1 - \left(\frac{s^\prime}{u}\right)^{\frac{1}{2k}}.
\end{split}
\end{equation*}

The number of copies cumulated while $p$ is rising from $p_0$ to $p^\ast$ can be calculated by integrating both sides:

\begin{equation*}
\begin{split}
	\frac{\mathrm d n}{\mathrm d p} &= \frac{k}{\pi}\left(1-\frac{s^\prime}{u(1-p)^{2k}}\right) \\
	\int_{n_0}^{n^\ast} \mathrm d n &=  \frac{k}{\pi} \left[\int_{p_0}^{p^\ast} \mathrm d p - \frac{s^\prime}{u} \int_{p_0}^{p^\ast} (1-p)^{-2k} \mathrm d p \right] \\
	n^\ast - n_0 &= \frac{k}{\pi}\left[p^\ast - p_0 - \frac{s^\prime}{u(2k-1)}((1-p^\ast)^{1-2k} - 1) \right] \\
	n^\ast &= n_0 + \frac{k}{\pi}\left[ p^\ast + \frac{s^\prime}{u(2k-1)}(1-(1-p^\ast)^{1-2k}) \right] \\
	n^\ast &= n_0 + \frac{k}{\pi}\left[ 1- \frac{1}{2k-1}\left(2k\left(\frac{s^\prime}{u}\right)^\frac{1}{2k} - \frac{s^\prime}{u}\right) \right].
\end{split}  
\end{equation*}

\subsection{\label{app:exhatp} Equation \ref{eq:exhatp}}

The strategy was very similar than for obtaining $n^\ast$, with $\mathrm d p / \mathrm d n$ integrated both sides from the maximum to the equilibrium:

\begin{equation*}
\begin{split}
	\int_{n^\ast}^{\hat n = 0} \mathrm d n &=  \frac{k}{\pi} \left[ \int_{p^\ast}^{\hat p} \mathrm d p - \frac{s^\prime}{u} \int_{p^\ast}^{\hat p} (1-p)^{-2k} \mathrm d p \right] \\
	-n^\ast &= \frac{k}{\pi}\left[(\hat p - p^\ast) - \frac{s^\prime}{u} \left( \frac{(1-\hat p)^{1-2k} - (1-p^\ast)^{1-2k}}{2k-1}\right)\right].
\end{split}
\end{equation*}

\subsection{\label{app:approxhatp} Equation \ref{eq:approxhatp}}

Rewriting the previous equation with $\delta p = \hat p - p^\ast$ and $1-p^\ast = q^\ast$ gives:

\begin{equation*}
	-n^\ast = \frac{k}{\pi}\left[\delta p - \frac{s^\prime}{u(1-2k)} \frac{1}{(q^\ast - \delta p)^{2k-1}} - \frac{s^\prime}{u(1-2k)} {q^\ast}^{1-2k}\right],
\end{equation*}

\noindent which turns out to be dominated by the second term ($ 1/(q^\ast - \delta p)^{2k-1} \gg \delta p$ when $\delta p$ increases) for most parameter values. As a consequence, neglecting $\hat p - p^\ast$ leads to:
\begin{equation*}
\begin{split}
	n^\ast &\simeq \frac{k}{\pi}\left[\frac{s^\prime}{u} \left( \frac{(1-\hat p)^{1-2k} - (1-p^\ast)^{1-2k}}{2k-1}\right)\right] \\
	\Longleftrightarrow \hat p &\simeq 1- \left[ (1-p^\ast)^{1-2k} + \frac{\pi u (2k-1)}{s^\prime k} n^\ast \right]^\frac{1}{1-2k}.
\end{split}
\end{equation*} 

Replacing $p^\ast$ and $n^\ast$ with their expressions from equation~\ref{eq:maxdtnc} and reorganizing gives:
\begin{equation*}
		\hat p \simeq 1-\left[\frac{u}{s^\prime}(2k-1)(1+\frac{n_0 \pi}{k}-\left(\frac{s^\prime}{u}\right)^\frac{1}{2k})+1\right]^\frac{1}{1-2k}.
\end{equation*}

Assuming that $n_0$ is reasonably small and $\pi \ll 1$, the term $n_0 \pi / k$ can be further neglected, and:
\begin{equation*}
		\hat p \simeq 1-\left[\frac{u}{s^\prime}(2k-1)(1-\left(\frac{s^\prime}{u}\right)^\frac{1}{2k})+1\right]^\frac{1}{1-2k}.
\end{equation*}

\subsection{\label{app:selpseq} Equilibrium stability for equation \ref{eq:selpseq}}

The Jacobian matrix corressponding to the equilibrum~\ref{eq:selpseq} (using the shortcuts $\dot n = \mathrm d n / \mathrm d t $ and  $\dot p = \mathrm d p / \mathrm d t$) is:

\begin{equation*}
\bm J = 
   \begin{bmatrix}  \frac{\partial \dot n}{\partial n} &  \frac{\partial \dot n}{\partial p} \\[1em]
					\frac{\partial \dot p}{\partial n} &  \frac{\partial \dot p}{\partial p} \end{bmatrix} =
   \begin{bmatrix*}[l]
				0 & -2 k \hat n u \left( \frac{s^\prime}{u}\right) ^\frac{2k-1}{2k} \\[1em]
				\frac{\pi s^\prime}{k} & -2 \hat n u \pi \left( \frac{s^\prime}{u}\right) ^\frac{2k-1}{2k} + \frac{1-s}{(1-s \hat p)^2} - 1
	\end{bmatrix*}.
\end{equation*}

Eigenvalues are negative (i.e., the equilibrum is stable) for all tested parameter combinations. Eigenvalues happen to be complex for all parameter combinations, except for very large values of $s$, the equilibrium is thus a stable focus, reached asymptotically by oscillating around it. 

\begin{figure}
\begin{adjustwidth}{-2in}{0in}
\begin{flushright}
	\includegraphics[width=15cm]{../figures/figF}
	\caption{\label{fig:figF}  First Eigenvalue of the Jacobian matrix as a function of model parameters ($u$, $s$, and $\pi$) in the deleterious TEs - deleterious cluster model. Default parameter values were $u=0.1$, $s=0.01$, and $\pi=0.03$. The number of clusters ($k=1$, $k=2$, and $k=5$) is indicated by different line styles. Eigenvalues are complex for most of the range of the parameters, real part is in black, imaginary part is in blue. }
\end{flushright}\end{adjustwidth}
\end{figure}

\end{appendices}

\end{document}
